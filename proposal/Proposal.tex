\documentclass{scrartcl}
\usepackage[utf8]{inputenc}
\usepackage[T1]{fontenc}
\usepackage{lmodern}
\DeclareUnicodeCharacter{00A0}{ }
\date{\today}
\title{Professional development}
\subtitle{Bouldering Wall}
\author{A. Dhar, J. Schloss, L. Li, V. Churavy}
\begin{document}
\maketitle
\section{Introduction}
Our project will concern itself with planning and building a bouldering wall for the OIST family.
Bouldering as a sport has seen a uptake of interest in recent years across Europe and North America, with creation of bouldering gyms in cities and at Universities. 

Bouldering is a form of sport climbing close to the ground and without the need of being belayed or wearing safety equipment like a harness, While bouldering the climber needs to focus on his balance on the wall, have the power and endurance to perform the next move, as well as a high degree of body control and foresight in planning his route.  Being a good climber does not mean being the strongest or fastest but rather being efficient and elegant. This mentality attracts a diverse set of people to the sport and is one of the main reasons for its rise in popularity.

A bouldering wall normally consists of several pieces of wood with different degrees of inclination on which the climbing holds are fixed. To limit the dangers related to falling the floor of the wall needs to be covered with foam.

\section{Construction of the Wall}
The wall itself consists of 12mm thick plywood that is reinforced by a frame made of wooden struts to carry the weight of the climber and to prevent flexing of the wood. The frame is the securely mounted to the base wall of the building. 
The foam mattress consists of a bottom layer of 20cm thick open foam and a top layer of 5cm closed foam, covered by a PVC sheet to protect the foam.
The two layer is necessary to prevent two distinct dangers that occur while climbing. The one is falling backwards of the wall and hurting you back and head in the process. The open foam layer is designed to absorb the force of the impact. The second danger is falling from the wall and landing on your feet. If the ground is too soft you might twist your ankle so the closed foam adds stability.

% Todo images

The first image is a look at the back of the wall, with the supporting frames and trusses needed to hold the wall together and attach it to the I-beams. The second image shows how the wall will fit into the corner of the ceramics shed, with three I-beams as its main supports and the foam pads underneath.

\section{Location}
As location for the wall we currently propose the old clay factory in which the student wood workshop is also located, but this is contingent on the approval of the building management and also the exact placement of the wall inside the clay factory needs to be determined. We would prefer one of the corners to be more efficient with the foam mattress while conserving space.

\section{Fixing the frame to the building}
Also contingent on the exact location is the question how to fix the wall to building structure and what kind of modifications we can make to it.

To attach the wall to the building, we have designed the wall to line up with three I-beams in the corner of the ceramics shed. The wall will be fixed to each of these beams with a brace made from metal pipes and brackets. The metal tube can be attached to the wall using U bolts, and then placed around each I-beam at the top and bottom, anchoring the wall securely. This also makes it possible for the wall to be removable should the need arise.

\section{Making Holds}
The climbing holds should be made out of Polyurethane in a variety of different forms. To create these forms, we will first design them with florist foam and cover them with silicone to produce a mold. After placing the Polyurethane in the mold, we will place the entire structure in a vacuum chamber. The chamber will consist of a table with a single (aperture) at the base, where toxic fumes will be siphoned through a (wetvac) to a safe, removed location. This process is necessary for curing because gas bubbles in the plastic will result in structural weaknesses. 

\section{Materials}
For obtaining hardware and supplies, our main sources will be Tabata and Makeman. We will also try to take advantage of the workshop’s supplies and tools when possible.
With our current design and our preferred location we would need the following materials:

\begin{itemize}
\item Frame
\begin{itemize}
\item 80m of 2x8cm struts 
\item 10m of 8x8cm struts
\item Screws + connectors
\end{itemize}
\item Wall
\begin{itemize}
\item 16.5 $m^2$ of 12mm plywood (910x1820mm pieces)
\item T-nuts 500 [20cm grid]
\end{itemize}
\item Mattress
\begin{itemize}
\item 6$m^2$ of 5cm thick closed foam
\item 6$m^2$ of 10 cm thick open foam (ICD 58/IFD>=60)
\item ~6$m^2$ pvc sheet
\end{itemize}
\item Misc
\begin{itemize}
\item Drill bits and bits
\end{itemize}
\item Holds
\begin{itemize}
\item Polyurethane resin
\item Florist foam
\item Silicone 
\item 10 mm bolts
\end{itemize}
\end{itemize}

\end{document}